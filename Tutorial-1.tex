\documentclass{article}
\usepackage[utf8]{inputenc}
\usepackage{titling}
\usepackage[margin=1in]{geometry}
\usepackage{amsmath}
\usepackage{tikz}
\usepackage{caption}
\usepackage{subcaption}
\usepackage[colorlinks=true, linkcolor=blue, urlcolor=blue]{hyperref}
\setlength{\parindent}{0pt}


\title{\large Discrete Structures (CS21201) Autumn 2025\\ \vspace{0.5em} \LARGE Tutorial 1}
\date{1st August 2025}

\begin{document}
\maketitle

\subsection*{Question 1:} Given a set of natural numbers ranging from 1 to n, the task at hand is to determine the count of subsets of size k that possess a distinctive characteristic: there are more odd numbers than even numbers.

\subsection*{Answer:}
Let $O$ be the number of odd elements in $\{1,2,\dots,n\}$, and $E$ be the number of even elements. Then:
\[
O = \left\lceil \frac{n}{2} \right\rceil, \quad E = \left\lfloor \frac{n}{2} \right\rfloor.
\]

We want to count the number of subsets of size $k$ that contain more odd numbers than even numbers.

Let $i$ be the number of odd elements in the subset. Since the total size is $k$, the number of even elements is $k - i$.

We require:
\[
i > k - i \quad \Rightarrow \quad 2i > k \quad \Rightarrow \quad i > \left\lfloor \frac{k}{2} \right\rfloor.
\]

So the number of such subsets is:
\[
\sum_{i = \left\lfloor \frac{k}{2} \right\rfloor + 1}^{\min(k, O)} \binom{O}{i} \binom{E}{k - i}.
\]

\subsection*{Question 2:} How many binary strings of length n are there with exactly k occurrences of the pattern 10? Assume that $n\ge 2k$. 

\subsection*{Answer:}
All such strings are of the form:
\[
\begin{array}{cccccccccccccccc}
1^{*} & 0^{*} & 01 & 1^{*} & 0^{*} & 01 & 1^{*} & 0^{*} & 01 & \cdots & 1^{*} & 0^{*} & 01 & 1^{*} & 0^{*} \\
x_1   & x_2   & 2  & x_3   & x_4   & 2  & x_5   & x_6   & 2  & \cdots & x_{2k-1} & x_{2k} & 2 & x_{2k+1} & x_{2k+2}
\end{array}
\]

So, we want to count the number of non-negative integer solutions to:
\[
x_1 + x_2 + 2 + x_3 + x_4 + 2 + x_5 + x_6 + 2 + \cdots + x_{2k-1} + x_{2k} + 2 + x_{2k+1} + x_{2k+2} = n,
\]
i.e.,
\[
x_1 + x_2 + \cdots + x_{2k+2} = n - 2k.
\]

So the number of solutions is:
\[
\binom{(n - 2k) + (2k + 2) - 1}{(2k + 2) - 1} = \binom{n + 1}{2k + 1}.
\]


\subsection*{Question 3:} Numbers ranging from 1 to 2n are arranged in a 2$\times$ n rectangular matrix, such that:
\begin{itemize}
    \item the elements in the columns are increasing. i.e. the element in the bottom row is greater than the cell above it.
    \item the elements in the row are in increasing order.
\end{itemize}

Example matrix (n = 5):
\[
\begin{bmatrix}
1 & 2 & 4 & 7 & 8 \\
3 & 5 & 6 & 9 & 10
\end{bmatrix}
\]

Compute the number of matrices satisfying the constraints in terms of n.

\subsection*{Answer:}

Any such matrix consists of a top row and a bottom row, together forming a partition of the numbers $1$ to $2n$.

\begin{itemize}
    \item Choose any $n$ of the $2n$ numbers to be in the top row. There are $\binom{2n}{n}$ ways to do this.
    \item Sort both the top and bottom row in increasing order (only one way to do so).
\end{itemize}

So, initially we have $\binom{2n}{n}$ candidate matrices with increasing rows.

Among the $\binom{2n}{n}$ matrices with increasing rows, we now want only those with increasing columns, i.e., the bottom entry in each column is greater than the top entry.

This is equivalent to requiring that the sequence formed by labeling each number as $U$ (if it's in the top row) or $D$ (if in the bottom row) forms a Dyck path: a balanced sequence of $n$ $U$'s and $n$ $D$'s, where no prefix has more $D$'s than $U$'s.

For each number in increasing order:
\begin{itemize}
    \item write $U$ if it's in the top row,
    \item write $D$ if it's in the bottom row.
\end{itemize}

This mapping is bijective as it is one-to-one (cell index to cardinality) and exhaustive over the range (as the number of cells is equal to the 2n). We get a word of length $2n$ with $n$ $U$'s and $n$ $D$'s.

The column-increasing condition ensures that the $i$-th $U$ appears before the $i$-th $D$ (otherwise some bottom entry would be smaller than the top one in a column). This exactly means that the word is a Dyck path.


The number of Dyck paths of length $2n$ (i.e., $n$ $U$'s and $n$ $D$'s, such that no prefix has more $D$'s than $U$'s) is exactly the $n$-th Catalan number:
\[
C_n = \frac{1}{n+1} \binom{2n}{n}
\]

\subsection*{Question 4:} Prove the following for all positive integers
\begin{equation*}
    \binom{2n}{n} = C(n) + \sum_{k=0}^{n-1} \binom{2n - 2k - 1}{n - k} \, C(k)
\end{equation*}

where $C(n)$ is the n'th Catalan number.

\subsection*{Answer:}
We consider all paths from $(0,0)$ to $(n,n)$ in the grid using only \texttt{R} (right) and \texttt{U} (up) moves. The total number of such paths is given by the binomial coefficient
\[
\binom{2n}{n},
\]
whereas $C(n)$ denotes the number of \emph{valid} Dyck paths — that is, paths that never cross the line $y = x$.

Thus, the sum on the right-hand side of the identity should count the number of \emph{invalid} paths from $(0,0)$ to $(n,n)$. Consider any such invalid path. By definition, it must cross the line $y = x$ at least once.

Let $k \in \{0, 1, 2, \dots, n - 1\}$ be the first instance where the path moves from $(k, k)$ to $(k, k+1)$ — that is, it steps \texttt{U} and goes above the line $y = x$ for the first time. Since this is the first such step, the portion of the path from $(0,0)$ to $(k,k)$ must be a valid Dyck path. The number of such paths is $C(k)$.

Now, from $(k, k)$ the path takes a \texttt{U} step to $(k, k+1)$. From there, we can choose any path to reach $(n,n)$. This is a path from $(k, k+1)$ to $(n,n)$, using $(n-k)$ right steps and $(n-k-1)$ up steps, which can be chosen in
\[
\binom{2n - 2k - 1}{n - k}
\]
ways.

Finally, summing over all allowed values of $k$ from $0$ to $n-1$, we obtain the number of invalid paths:
\[
\binom{2n}{n} = C(n) + \sum_{k=0}^{n-1} \binom{2n - 2k - 1}{n - k} \cdot C(k).
\]

\subsection*{Question 5:} You are given an n×n grid. You want to count the number of rectangles in the grid, in which the horizontal sides are larger than the vertical sides. The figure \ref{fig:1} shows the 11 possible such rectangles in a 3×3 grid.

\begin{figure}[h!]
\centering

\newcommand{\gridrect}[4]{%
\begin{subfigure}{0.085\textwidth}
\centering
\begin{tikzpicture}[scale=0.4]
    \foreach \x in {0,1,2,3} \draw (\x,0) -- (\x,3);
    \foreach \y in {0,1,2,3} \draw (0,\y) -- (3,\y);
    \fill[gray] (#1,#2) rectangle (#3,#4);
\end{tikzpicture}
\end{subfigure}
}

\gridrect{0}{2}{2}{3}
\gridrect{1}{2}{3}{3}
\gridrect{0}{1}{2}{2}
\gridrect{1}{1}{3}{2}
\gridrect{0}{0}{2}{1}
\gridrect{1}{0}{3}{1}
\vspace{0.4cm}

\gridrect{0}{2}{3}{3}
\gridrect{0}{1}{3}{2}
\gridrect{0}{0}{3}{1}
\vspace{0.4cm}

\gridrect{0}{1}{3}{3}
\gridrect{0}{0}{3}{2}
\caption{All 11 rectangles in a $3 \times 3$ grid where width $>$ height}
\label{fig:1}
\end{figure}

Deduce a closed-form expression for the desired count (as a function of n).

\subsection*{Answer:}
Let $R_{>}, R_{=}, R_{<}$ denote the numbers of rectangles in an $n \times n$ grid whose horizontal sides are, respectively, larger than, equal to, and smaller than the vertical sides. Then the total number of rectangles is

\[
R = R_{>} + R_{=} + R_{<}.
\]

(Note: Squares are also treated as rectangles.)

A rectangle in the grid is fully specified by choosing:

- the coordinates $(i, j)$ of the bottom-left corner, and  
- the coordinates $(k, \ell)$ of the top-right corner,

such that:
\[
0 \leq i < k \leq n \quad \text{and} \quad 0 \leq j < \ell \leq n.
\]

The number of ways to choose $(i, k)$ with $i < k$ is $\binom{n+1}{2}$, and similarly for $(j, \ell)$. Therefore, the total number of rectangles is:

\[
R = \binom{n+1}{2}^2 = \left( \frac{n(n+1)}{2} \right)^2 = \frac{n^2(n+1)^2}{4}.
\]


There are:

- $n^2$ squares of side 1,  
- $(n-1)^2$ squares of side 2,  
- $(n-2)^2$ squares of side 3,  
- $\dots$,  
- $1^2$ square of side $n$.

So the total number of squares is:

\[
R_{=} = \sum_{k=1}^{n} k^2 = \frac{n(n+1)(2n+1)}{6}.
\]


By symmetry, the number of rectangles with width $>$ height equals the number with width $<$ height. Therefore:

\[
R_{>} = R_{<} = \frac{1}{2}(R - R_{=}).
\]

Substitute the expressions for $R$ and $R_{=}$:

\begin{align*}
R_{>} &= \frac{1}{2} \left( \frac{n^2(n+1)^2}{4} - \frac{n(n+1)(2n+1)}{6} \right) \\
&= \frac{n(n+1)}{2} \left( \frac{3n(n+1) - 2(2n+1)}{12} \right) \\
&= \frac{n(n+1)(n-1)(3n+2)}{24}
\end{align*}

\vspace{0.5cm}
\noindent
\textbf{Extras:}\\
\href{https://www.mat.univie.ac.at/~slc/wpapers/FPSAC2017/39%20Borie.pdf}{3-D Catalan Numbers}\\
Practice problems from Ralph Grimaldi
\end{document}
